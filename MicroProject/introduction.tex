\chapter*{Introduction}
\label{chapter:introduction}

The purpose of this report was to evaluate the knowledge of antenna simulations and general competence in MATLAB.  This was illustrated by computing the far field functions $\mathbf{G}(\theta, \phi)$, radiated power $P_{rad}$, radiation patterns, far field patterns, array factor patterns and excitation currents $\{I_n\}$ of different types of antennas in MATLAB. 

Two kinds of antennas were considered. In chapters \ref{chapter:task1}-\ref{chapter:task3} a horizontal dipole antenna was simulated in MATLAB where first the far field of the dipole antenna was computed as a function of observation angles and the excitation current. Then the total radiated power was computed as a function of the far field, then the far field patterns  and co polar radiation patterns evaluated as a function of the far field and observation angles in the E- ad H-plane were computed and plotted.

 In chapters \ref{chapter:task4}- \ref{chapter:task6} the second type of antenna to be considered was array antennas. First the far field function was computed for an irregular array antenna as a function of element position and excitation current. Then the excitation current was computed, as a function of scanning angles and element positions for an irregularly spaced array antenna. The amplitude of this current was set to unity. Lastly an equispaced array was considered and the far field, the radiation patterns were computed in the H-plane and broadside as a function of the far field of the array antenna and observation angles. 
 
 I chapter \ref{chapter:discussion} the results from the dipole and array antennas will be discussed, in appendix A the complete code written in MATLAB may be viewed by the reader.  