\chapter{Task 2}
\label{chapter:task2}
In this chapter the total radiated power was computed as a function of the far field for the horizontal dipole, with a chosen polarization, that was considered in chapter \ref{chapter:task1}. Initially the theory regarding the radiated power will be presented, then the results from the computations will be shown. 

\section{Theory}
The total radiated power can be evaluated by considering the co- and cross-polarizations, which are defined as 
\begin{equation}
\hat{co} = cos(\phi - \epsilon) \hat{\theta} - sin(\phi - \epsilon)\hat{\phi}
\end{equation} 
and
\begin{equation}
\hat{xp} = -sin(\phi - \epsilon) \hat{\theta} - cos(\phi - \epsilon)\hat{\phi}
\end{equation}
Where $\epsilon$ can be chosen to achieve the different polarizations. In order to achieve a linear  x polarization we set $\epsilon$ = 0 which gives us 
\begin{align}
&\hat{co} = cos(\phi)\hat{\theta} - sin(\phi)\hat{\phi} \\
&\hat{co} = -sin(\phi)\hat{\theta} - cos(\phi)\hat{\phi}.
\end{align}
To get the y polarization $\epsilon$ is set to $\pi/2$, giving the expression 
\begin{align}
&\hat{co} = cos(\phi-\pi/2)\hat{\theta} - sin(\phi-\pi/2)\hat{\phi} \\
&\hat{co} = -sin(\phi-\pi/2)\hat{\theta} - cos(\phi-\pi/2)\hat{\phi}.
\end{align}
For RHC we polarization vectors become  
\begin{align}
&\hat{co} = e^{-j\phi}/\sqrt{2}(\hat{\theta} - j\hat{\phi}) \\
&\hat{xp} = e^{j\phi}/\sqrt{2}(\hat{\theta} + j)\hat{\phi}
\end{align}
and for LHC
\begin{align}
&\hat{co} = e^{j\phi}/\sqrt{2}(\hat{\theta} + j\hat{\phi}) \\
&\hat{xp} = e^{-j\phi}/\sqrt{2}(\hat{\theta} - j)\hat{\phi}.
\end{align}
The total radiated power is defined to be 
\begin{equation}
P_{rad} = \int \int_{\text{far field sphere}}(\langle\mathbf{W}\rangle \cdot \hat{r})dA
\end{equation}
which can be rewritten in spherical coordinates giving the expression 
\begin{equation}
P_{rad} = \int_{0}^{2\pi} \int_{0}^{\pi} (\langle\mathbf{W}\rangle \cdot \hat{r})r^2sin\theta d\theta d\phi.
\end{equation}
The total radiated power can then be computed by integrating over the sphere of the squared norm of the far , i.e. 
\begin{equation}
P_{rad} =\frac{1}{2 \eta} \int_0^{2\pi} d\phi \int_0^{\pi} d\theta sin(\theta)(|G_{co}|^2 + |G_{xp}|^2).
\end{equation}
$\eta$ can be set to $120\pi \Omega$ is room temperature. 
The integration can easily be calculated with the trapez  method or the midpoint method in order to get a numerical result. In order to get the co- and cross- polar parts of the far field function we simply multiply it with the conjugate of the co- and cross- polar vectors, i.e.
\begin{align}
& G_{co}  = \mathbf{G}\cdot \hat{co}^* \\
& G_{xp}  = \mathbf{G}\cdot \hat{xp}^*.
\end{align}
Where we must choose a specific polarization in order to get the numerical results.\cite{kildal2000foundations}
   

\section{Results}
The MATLAB code for this task can be viewed in \ref{section:task2.m}. The numerical result of the total radiated power was 9.5984 W, with y, x, LHC and RHC polarization. The excitation current was chosen to be unity A for this task.


